%! Author = mishi9
%! Date = 24-09-20

The goal is to make a sensor that measures alcohol concentration in a liquid.
Water and alcohol have different physical attributes, and by measuring the liquid's properties we can determine what mixture of water to alcohol it contains.
Such sensors already exist by measuring the buoyancy of the liquid to determine the mixture proportions.
However, such sensors involve more mechanics than electronics.
Therefore, the sensor system will work on a different principle - the liquid's dielectric constant.

A software defined radio (SDR) will be used to transmit radio signals of a varying frequency from an antenna that is submerged in the tested liquid.
Another antenna will be attached to the SDR and it will be used to receive the transmitted radio signals.
From the attenuation and phase shift of the radio signals caused by the liquid, its relative permittivity can be determined and thus its dielectric constant.
This dielectric constant can be used to determine what the ratio of alcohol to water is, as the permittivity is a function of the proportion of alcohol to water in the liquid.
The calculations will be done in software on the PC that is connected to the SDR.